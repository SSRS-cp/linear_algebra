\documentclass{jlreq}
\usepackage{amsfonts}
\usepackage{amsmath}
\usepackage{amsthm}
\usepackage{bm}
\usepackage{enumitem}
\usepackage{verbatim}
\title{線形代数}
\author{}
\setlength{\parindent}{0pt}
\everymath{\displaystyle}
\theoremstyle{definition}
\newtheorem{thm}{定理}[subsection]
\newtheorem{dfn}[thm]{定義}
\newtheorem{sym}[thm]{記号}
\newtheorem{prop}[thm]{命題}
\newtheorem{lem}[thm]{補題}
\newtheorem{rem}[thm]{注}
\newtheorem{cor}[thm]{系}
\newtheorem{ex}[thm]{例}
\newcommand{\relmiddle}[1]{\mathrel{}\middle#1\mathrel{}}
\DeclareMathOperator{\Ker}{Ker}
\DeclareMathOperator{\Img}{Im}
\begin{document}
  \maketitle
  \tableofcontents
  \newpage
  \section{ベクトル空間}
    \subsection{ベクトル空間}
      \begin{dfn}
        $K$ を体とする.集合 $V$ に加法 $V^2 \ni (\bm{x}_1,\bm{x}_2) \mapsto \bm{x}_1+\bm{x}_2 \in V$ とスカラー倍 $K \times V \ni (\lambda,\bm{x}) \mapsto \lambda\bm{x} \in V$ が定義され,以下の性質を満たすとき,$V$ を $K$ 上のベクトル空間 (あるいはベクトル空間) という.また,このとき $V$ の元をベクトルという.
        \begin{enumerate}
          \item $\bm{x_1}+\bm{x_2}=\bm{x_2}+\bm{x_1}$.
          \item $(\bm{x_1}+\bm{x_2})+\bm{x_3}=\bm{x_1}+(\bm{x_2}+\bm{x_3})$.
          \item ある $V$ の元 $\bm{0}$ が存在し,$\bm{x}+\bm{0}=\bm{0}+\bm{x}=\bm{x}$.$\bm{0}$ を零ベクトルという.
          \item 各 $\bm{x} \in V$ に対し,$\bm{x}' \in V$ が存在し,$\bm{x}+\bm{x}'=\bm{x}'+\bm{x}=\bm{0}$.この $\bm{x}'$ を $\bm{x}'$ の逆ベクトルという.
          \item $(\lambda_1+\lambda_2)\bm{x}=\lambda_1\bm{x}+\lambda_2\bm{x}$.
          \item $\lambda(\bm{x}_1+\bm{x}_2)=\lambda\bm{x}_1+\lambda\bm{x}_2$.
          \item $(\lambda_1\lambda_2)\bm{x}=\lambda_1(\lambda_2\bm{x})$.
          \item $1\bm{x}=\bm{x}$.
        \end{enumerate}
      \end{dfn}
      \begin{sym}
        $\bm{x} \in V$ の逆ベクトル $\bm{x}'$ を $-\bm{x}$ とも書く.また,$\bm{x}_1,\bm{x}_2 \in V$ に対し $\bm{x}_1+(-\bm{x}_2)$ を $\bm{x}_1-\bm{x}_2$ とも書く.
      \end{sym}
      \begin{comment}
      \end{comment}
      \begin{prop}
        $V$: ベクトル空間.
        \begin{enumerate}
          \item 零ベクトル $\bm{0}$ はただ $1$ つ存在する.
          \item $\bm{x} \in V$ に対し,$\bm{x}$ の逆ベクトルはただ $1$ つ存在する.
          \item $0\bm{x}=\bm{0}$.
          \item $\lambda\bm{0}=\bm{0}$.
          \item $(-1)\bm{x}=-\bm{x}$.
        \end{enumerate}
      \end{prop}
      \begin{proof}
        \mbox{}
        \begin{enumerate}
          \item $\bm{0}_1, \bm{0}_2 \in V$ を $V$ の零ベクトルとすると,$\bm{0}_1 = \bm{0}_1+\bm{0}_2 = \bm{0}_2$.
          \item $\bm{x}'_1, \bm{x}'_2 \in V$ を $\bm{x} \in V$ の逆ベクトルとすると,$\bm{x}'_1 = \bm{x}'_1+\bm{0} = \bm{x}'_1+(\bm{x}+\bm{x}'_2) = (\bm{x}'_1+\bm{x})+\bm{x}'_2 = \bm{0}+\bm{x}'_2 = \bm{x}'_2$.
          \item $0\bm{x} = 0\bm{x}+\bm{0} = 0\bm{x}+(0\bm{x}-0\bm{x})=(0\bm{x}+0\bm{x})-0\bm{x}=(0+0)\bm{x}-0\bm{x}=0\bm{x}-0\bm{x}=\bm{0}$.
          \item $\lambda\bm{0}=\lambda\bm{0}+\bm{0}=\lambda\bm{0}+(\lambda\bm{0}-\lambda\bm{0})=(\lambda\bm{0}+\lambda\bm{0})-\lambda\bm{0}=\lambda(\bm{0}+\bm{0})-\lambda\bm{0}=\lambda\bm{0}-\lambda\bm{0}=\bm{0}$.
          \item $\bm{x}+(-1)\bm{x}=1\bm{x}+(-1)\bm{x}=(1+(-1))\bm{x}=0\bm{x}=\bm{0}, (-1)\bm{x}+\bm{x}=(-1+1)\bm{x}=0\bm{x}=\bm{0}$ より,$(-1)\bm{x}$ は $\bm{x}$ の逆ベクトルである.
        \end{enumerate}
      \end{proof}
      \begin{prop}
        $V$: ベクトル空間. 
        \begin{enumerate}
          \item $-\bm{0}=\bm{0}$.
          \item $-(-\bm{x})=\bm{x}$.
          \item $-(\bm{x}_1+\bm{x}_2)=-\bm{x}_1-\bm{x}_2$.
          \item $-(\lambda\bm{x})=(-\lambda)\bm{x}$.
        \end{enumerate}
      \end{prop}
      \begin{proof}
        \mbox{}
        \begin{enumerate}
          \item $-\bm{0}=(-1)\bm{0}=\bm{0}$.
          \item $-(-\bm{x})=-((-1)\bm{x})=(-1)((-1)\bm{x})=((-1)(-1))\bm{x}=1\bm{x}=\bm{x}$.
          \item $-(\bm{x}_1+\bm{x}_2)=(-1)(\bm{x}_1+\bm{x}_2)=(-1)\bm{x}_1+(-1)\bm{x}_2=-\bm{x}_1-\bm{x}_2$.
          \item $-(\lambda\bm{x})=(-1)(\lambda\bm{x})=((-1)\lambda)\bm{x}=(-\lambda)\bm{x}$.
        \end{enumerate}
      \end{proof}
      \begin{dfn}
        $V$: ベクトル空間.
        $W \subseteq V$ が $V$ と同じ和とスカラー倍によりベクトル空間となるとき,$W$ は $V$ の部分空間であるという.
      \end{dfn}
      \begin{prop}
        $V$: ベクトル空間.
        \begin{enumerate}
          \item $\{\bm{0}\} \subseteq V$ は $V$ の部分空間である.
          \item $V \subseteq V$ は $V$ の部分空間である.
        \end{enumerate}
      \end{prop}
      \begin{proof}
        \mbox{}
        自明.
      \end{proof}
      \begin{lem}\label{zero-vector-and-inverse-are-in-subspace}
        $V$: ベクトル空間,$W \subset V$: 部分空間.
        \begin{enumerate}
          \item $V$ の零ベクトルを $\bm{0}$ とおくと,$\bm{0} \in W$ である.また,$\bm{0}$ は $W$ の零ベクトルである.
          \item $\bm{x} \in W$ について,$V$ における $\bm{x}$ の逆ベクトルを $\bm{x}'$ とおくと,$\bm{x}' \in W$ である.また,$\bm{x}'$ は $W$ における $\bm{x}$ の逆ベクトルである.
        \end{enumerate}
      \end{lem}
      \begin{proof}
        \mbox{}
        \begin{enumerate}
          \item $W \neq \emptyset$ なので,$\bm{x} \in W$ を取ると $\bm{x}-\bm{x}=\bm{0}$.一方,$\bm{x}-\bm{x} \in W$ なので $\bm{0} \in W$.また,$\forall \bm{x} \in W$ に対し $\bm{x}+\bm{0}=\bm{0}+\bm{x}=\bm{x}$ なので $\bm{0}$ は $W$ の零ベクトルである.
          \item $W$ における $\bm{x}$ の逆ベクトルを $\bm{x}''$ とおくと,$\bm{0}$ は $W$ の零ベクトルなので,$\bm{x}+\bm{x}''=\bm{x}''+\bm{x}=\bm{0}$ であるから $\bm{x}''$ は $\bm{x}'$ に等しい.
        \end{enumerate}
      \end{proof}
      \begin{prop}
        $V$: $K$ 上のベクトル空間,$W \subseteq V$.以下は同値.
        \begin{enumerate}
          \item $W$ は $V$ の部分空間である.
          \item 以下の (a), (b), (c) が成り立つ.
            \begin{enumerate}
              \item $W \neq \emptyset$.
              \item $\forall \bm{x}_1,\bm{x}_2 \in W: \bm{x}_1+\bm{x}_2 \in W$.
              \item $\forall \lambda \in K, \bm{x} \in W: \lambda\bm{x} \in W$.
            \end{enumerate}
          \item 以下の (a), (b) が成り立つ.
            \begin{enumerate}
              \item $W \neq \emptyset$.
              \item $\forall \lambda_1,\lambda_2 \in K, \bm{x}_1,\bm{x}_2 \in W: \lambda_1\bm{x}_1+\lambda_2\bm{x}_2 \in W$.
            \end{enumerate}
        \end{enumerate}
      \end{prop}
      \begin{proof}
        \mbox{}
        \begin{description}
          \item [$(1. \implies 3.) \, $] 自明.
          \item [$(3. \implies 2.) \, $] \mbox{}\begin{enumerate}[label=\text{(\alph*) }] 
              \item 自明.
              \item $\bm{x}_1+\bm{x}_2=1\bm{x}_1+1\bm{x}_2 \in W$.
              \item $\lambda \bm{x} = \lambda\bm{x}+0\bm{x} \in W$.
          \end{enumerate}
          \item [$(2. \implies 1.) \, $]
            \mbox{}
            
            $W$ は加法とスカラー倍について閉じているので,定義域を $W$ に制限することにより $W$ に $V$ と同じ加法 $W \ni (\bm{x}_1,\bm{x}_2) \mapsto \bm{x}_1,\bm{x}_2 \in W$ とスカラー倍 $K \times V \ni (\lambda,\bm{x}) \mapsto \lambda\bm{x} \in V$ が定義できる.補題 \ref{zero-vector-and-inverse-are-in-subspace} より,$W$ はベクトル空間となる.
        \end{description}
      \end{proof}
      \begin{prop}
        $V$: $K$ 上のベクトル空間,$W_1, W_2, \dots, W_n \subseteq V$: $V$ の部分空間.このとき,$\bigcap_{i=1}^n W_i := W_1 \cap W_2 \cap \dots \cap W_n \subseteq V$ は $V$ の部分空間.
      \end{prop}
      \begin{proof}
        $\bm{0} \in W_i \, (i \in \{1,2,\dots,n\})$ より,$\bm{0} \in \bigcap_{i=1}^n W_i$ なので $\bigcap_{i=1}^n W_i \neq \emptyset$.

        $\lambda_1,\lambda_2 \in K, \bm{x}_1,\bm{x}_2 \in \bigcap_{i=1}^n W_i$ とおく.$i \in \{1,2,\dots,n\}$ に対し,$\bm{x}_1,\bm{x}_2 \in W_i$ より $\lambda_1\bm{x}_1+\lambda_2\bm{x}_2 \in W_i$ であるから,$\lambda_1\bm{x}_1+\lambda_2\bm{x}_2 \in \bigcap_{i=1}^n W_i$.
      \end{proof}
      \begin{prop}
        $V$: ベクトル空間,$W_1, W_2, \dots, W_n \subseteq V$: $V$ の部分空間.このとき,$\sum_{i=1}^n = W_1 + W_2 + \dots + W_n := \left\{\sum_{i=1}^n\bm{x}_i \relmiddle| \bm{x}_i \in W_i\right\} \subseteq V$ は $V$ の部分空間.
      \end{prop}
      \begin{proof}
        $\bm{0} \in W_i \, (i \in \{1,2,\dots,n\})$ より,$\sum_{i=1}^n\bm{0} \in \sum_{i=1}^n W_i$ なので $\sum_{i=1}^n W_i \neq \emptyset$.

        $\lambda_1,\lambda_2 \in K, \bm{x}_1,\bm{x}_2 \in \sum_{i=1}^n W_i$ とおくと,$\exists \bm{x}_{i,j} \in W_j s.t. \bm{x}_i=\sum_{j=1}^n\bm{x}_{i,j} \, (i \in \{1,2\}, j \in \{1,2,\dots,n\})$.このとき,$\lambda_1\bm{x}_1+\lambda_2\bm{x}_2=\sum_{j=1}^n(\lambda_1\bm{x}_{1,j}+\lambda_2\bm{x}_{2,j})$ であり,$\lambda_1\bm{x}_{1,j}+\lambda_2\bm{x}_{2,j} \in W_j$ なので $\lambda_1\bm{x}_1+\lambda_2\bm{x}_2 \in \sum_{i=1}^n W_i$.
      \end{proof}
    \subsection{線形写像}
      \begin{dfn}
        $V_1$, $V_2$: $K$ 上のベクトル空間.写像 $T \colon V_1 \to V_2$ が以下の条件を満たすとき,$T$ は線形写像であるという.
        \begin{enumerate}
          \item $T(\bm{x}_1+\bm{x}_2)=T(\bm{x}_1)+T(\bm{x}_2) \, (\bm{x}_1,\bm{x}_2 \in V_1)$.
          \item $T(\lambda\bm{x})=\lambda T(\bm{x}) \, (\lambda \in K, \bm{x} \in V_1)$.
        \end{enumerate}
      \end{dfn}
      \begin{sym}
        線形写像 $V_1 \to V_2$ 全体の集合を $\mathcal{L}(V_1,V_2)$ と書く.また,$\mathcal{L}(V,V)$ を $\mathcal{L}(V)$ と書く.
      \end{sym}
      \begin{prop}
        $V_1, V_2$: $K$ 上のベクトル空間.写像 $T \colon V_1 \to V_2$ について以下は同値.
        \begin{enumerate}
          \item $T \in \mathcal{L}(V_1,V_2)$.
          \item $\forall \lambda_1,\lambda_2 \in K, \bm{x}_1,\bm{x}_2 \in V_1$ について,$T(\lambda_1\bm{x}_1+\lambda_2\bm{x}_2)=\lambda_1T(\bm{x}_1)+\lambda_2T(\bm{x}_2)$.
        \end{enumerate}
      \end{prop}
      \begin{proof}
        \mbox{}
        \begin{description}
          \item [$(1. \implies 2.) \, $]
            \mbox{}
            
            $T \in \mathcal{L}(V_1,V_2)$ とすると,$\lambda_1,\lambda_2 \in K, \bm{x}_1,\bm{x}_2 \in V_1$ に対し,$T(\lambda_1\bm{x}_1+\lambda_2\bm{x}_2)=T(\lambda_1\bm{x}_1)+T(\lambda_2\bm{x}_2)=\lambda_1T(\bm{x}_1)+\lambda_2T(\bm{x}_2)$.
          \item [$(2. \implies 1.) \, $]
            \mbox{}
            
            $\bm{x}_1,\bm{x}_2 \in V_1$ に対し,$T(\bm{x}_1+\bm{x}_2)=T(1\bm{x}_1+1\bm{x}_2)=1T(\bm{x}_1)+1T(\bm{x}_2)=T(\bm{x}_1)+T(\bm{x}_2)$.また,$\lambda \in K, \bm{x} \in V_1$ に対し,$T(\lambda \bm{x}) = T(\lambda \bm{x} + 0\bm{x}) = \lambda T(\bm{x}) + 0T(\bm{x}) = \lambda T(\bm{x})$.よって $T\in\mathcal{L}(V_1,V_2)$.
        \end{description}
      \end{proof}
      \begin{prop}
        $V_1, V_2$: ベクトル空間,$T \in \mathcal{L}(V_1, V_2)$.このとき,$T(\bm{0})=\bm{0}$.
      \end{prop}
      \begin{proof}
        $T(\bm{0})=T(0\bm{0})=0T(\bm{0})=\bm{0}$.
      \end{proof}
      \begin{ex}
        $V$: $K$ 上のベクトル空間.写像 $V \ni \bm{x} \mapsto \bm{x} \in V$ は線形写像である.この写像を $V$ における恒等写像といい,$1_V$ と表す.
      \end{ex}
      \begin{proof}
        $\lambda_1,\lambda_2 \in K, \bm{x}_1,\bm{x}_2 \in V$ に対し,$1_V(\lambda_1\bm{x}_1+\lambda_2\bm{x}_2)=\lambda_1\bm{x}_1+\lambda_2\bm{x}_2=\lambda_11_V(\bm{x}_1)+\lambda_21_V(\bm{x}_2)$.
      \end{proof}
      \begin{ex}
        $V_1,V_2$: $K$ 上のベクトル空間.
        \begin{enumerate}
          \item 写像 $V_1 \ni \bm{x} \mapsto \bm{0} \in V_2$ は線形写像である.この写像を $0$ 写像といい,$0 \in \mathcal{L}(V_1,V_2)$ と表す.
          \item $T \in \mathcal{L}(V_1,V_2)$ に対し,写像 $V_1 \ni \bm{x} \mapsto -T(\bm{x}) \in V_2$ は線形写像である.この写像を $-T \in \mathcal{L}(V_1,V_2)$ と表す.
        \end{enumerate}
      \end{ex}
      \begin{proof}
        \mbox{}
        \begin{enumerate}
          \item $\lambda_1,\lambda_2 \in K, \bm{x}_1,\bm{x}_2 \in V_1$ に対し,$0(\lambda_1\bm{x}_1+\lambda_2\bm{x}_2)=\bm{0}=\lambda_1\bm{0}+\lambda_2\bm{0}=\lambda_10(\bm{x}_1)+\lambda_20(\bm{x}_2)$.
          \item $\lambda_1,\lambda_2 \in K, \bm{x}_1,\bm{x}_2 \in V_1$ に対し,
            \begin{align*}
              (-T)(\lambda_1\bm{x}_1+\lambda_2\bm{x}_2) &= -(T(\lambda_1\bm{x}_1+\lambda_2\bm{x}_2)) \\
              &= -(\lambda_1T(\bm{x}_1)+\lambda_2T(\bm{x}_2)) \\
              &= -\lambda_1T(\bm{x}_1)-\lambda_2T(\bm{x}_2) \\
              &= \lambda_1(-T(\bm{x}_1))+\lambda_2(-T(\bm{x}_2)) \\
              &= \lambda_1(-T)(\bm{x}_1)+\lambda_2(-T)(\bm{x}_2).
            \end{align*}
        \end{enumerate}
      \end{proof}
      \begin{dfn}\label{linear-map-addition-and-scalar-multiplication}
        $V_1, V_2$: $K$ 上のベクトル空間.
        \begin{enumerate}
          \item $T_1,T_2\in\mathcal{L}(V_1,V_2)$ に対し,写像 $T_1+T_2 \colon V_1 \to V_2$ を $T_1+T_2 \colon V_1 \ni \bm{x} \mapsto T_1(\bm{x})+T_2(\bm{x}) \in V_2$ と定める.
          \item $T \in \mathcal{L}(V_1,V_2), \lambda \in K$ に対し,写像 $\lambda T \colon V_1 \to V_2$ を $\lambda T \colon V_1 \ni \bm{x} \mapsto \lambda(T(x)) \in V_2$ と定める.
        \end{enumerate}
      \end{dfn}
      \begin{prop}
        $V_1, V_2$: $K$ 上のベクトル空間.
        \begin{enumerate}
          \item $T_1,T_2 \in \mathcal{L}(V_1,V_2)$ のとき,$T_1+T_2 \in \mathcal{L}(V_1,V_2)$.
          \item $T\in\mathcal{L}(V_1,V_2), \lambda\in K$ のとき,$\lambda T \in \mathcal{L}(V_1,V_2)$.
        \end{enumerate}
      \end{prop}
      \begin{proof}
        \mbox{}
        \begin{enumerate}
          \item $\lambda_1,\lambda_2 \in K, \bm{x}_1,\bm{x}_2 \in V_1$ に対し,
            \begin{align*}
              (T_1+T_2)(\lambda_1\bm{x}_1+\lambda_2\bm{x}_2) &= T_1(\lambda_1\bm{x}_1+\lambda_2\bm{x}_2)+T_2(\lambda_1\bm{x}_1+\lambda_2\bm{x}_2) \\
              &= \lambda_1T_1(\bm{x}_1)+\lambda_2T_1(\bm{x}_2)+\lambda_1T_2(\bm{x}_1)+\lambda_2T_2(\bm{x}_2) \\
              &= \lambda_1T_1(\bm{x}_1)+\lambda_1T_2(\bm{x}_1)+\lambda_2T_1(\bm{x}_2)+\lambda_2T_2(\bm{x}_2) \\
              &= \lambda_1(T_1(\bm{x}_1)+T_2(\bm{x}_1))+\lambda_2(T_1(\bm{x}_2)+T_2(\bm{x}_2)) \\
              &= \lambda_1(T_1+T_2)(\bm{x}_1)+\lambda_2(T_1+T_2)(\bm{x}_2).
            \end{align*}
          \item $\lambda_1,\lambda_2 \in K, \bm{x}_1,\bm{x}_2 \in V_1$ に対し,
            \begin{align*}
              (\lambda T)(\lambda_1\bm{x}_1+\lambda_2\bm{x}_2) &= \lambda(T(\lambda_1\bm{x}_1+\lambda_2\bm{x}_2)) \\
              &= \lambda(\lambda_1T(\bm{x}_1)+\lambda_2T(\bm{x}_2)) \\
              &= \lambda(\lambda_1T(\bm{x}_1))+\lambda(\lambda_2T(\bm{x}_2)) \\
              &= (\lambda\lambda_1)T(\bm{x}_1)+(\lambda\lambda_2)T(\bm{x}_2) \\
              &= (\lambda_1\lambda)T(\bm{x}_1)+(\lambda_2\lambda)T(\bm{x}_2) \\
              &= \lambda_1(\lambda T(\bm{x}_1))+\lambda_2(\lambda T(\bm{x}_2)) \\
              &= \lambda_1(\lambda T)(\bm{x}_1)+\lambda_2(\lambda T)(\bm{x}_2).
            \end{align*}
        \end{enumerate}
      \end{proof}
      \begin{prop}
        $V_1,V_2$: $K$ 上のベクトル空間.このとき,$\mathcal{L}(V_1,V_2)$ は定義
         \ref{linear-map-addition-and-scalar-multiplication} で定めた加法とスカラー倍について $K$ 上のベクトル空間となる.
      \end{prop}
      \begin{proof}
        \mbox{}
        \begin{enumerate}
          \item $T_1,T_2 \in \mathcal{L}(V_1,V_2)$ とおくと,$\bm{x} \in V_1$ に対し,
            \begin{align*}
              (T_1+T_2)(x) &= T_1(x)+T_2(x) \\
              &= T_2(x)+T_1(x) \\
              &= (T_2+T_1)(x)
            \end{align*}
            より,$T_1+T_2=T_2+T_1$.
          \item $T_1,T_2,T_3 \in \mathcal{L}(V_1,V_2)$ とおくと,$\bm{x} \in V_1$ に対し,
            \begin{align*}
              ((T_1+T_2)+T_3)(x) &= (T_1+T_2)(x)+T_3(x) \\
              &= (T_1(x)+T_2(x))+T_3(x) \\
              &= T_1(x)+(T_2(x)+T_3(x)) \\
              &= T_1(x)+(T_2+T_3)(x) \\
              &= (T_1+(T_2+T_3))(x)
            \end{align*}
            より,$(T_1+T_2)+T_3=T_1+(T_2+T_3)$.
          \item $T \in \mathcal{L}(V_1,V_2), \bm{x} \in V_1$ に対し,
            \begin{align*}
              (0+T)(\bm{x}) &= 0(\bm{x})+T(\bm{x}) \\
              &= \bm{0}+T(\bm{x}) \\
              &= T(\bm{x}),\\
              (T+0)(\bm{x}) &= T(\bm{x})+0(\bm{x}) \\
              &= T(\bm{x})+\bm{0} \\
              &= T(\bm{x})
            \end{align*}
            より,$0+T=T+0=T$ である.
          \item $T \in \mathcal{L}(V_1,V_2), \bm{x} \in V_1$ に対し,
            \begin{align*}
              (T+(-T))(\bm{x}) &= T(\bm{x})+(-T)(\bm{x}) \\
              &= T(\bm{x})-T(\bm{x}) \\
              &= \bm{0} \\
              &= 0(\bm{x}),\\
              (-T+T)(\bm{x}) &= (-T)(\bm{x})+T(\bm{x}) \\
              &= -T(\bm{x})+T(\bm{x}) \\
              &= \bm{0} \\
              &= 0(\bm{x})
            \end{align*}
            より,$T+(-T)=(-T)+T=0$ である.
          \item $\lambda_1,\lambda_2 \in K, T \in \mathcal{L}(V_1,V_2)$ とおくと,$\bm{x} \in V_1$ に対し,
            \begin{align*}
              ((\lambda_1+\lambda_2)T)(\bm{x}) &= (\lambda_1+\lambda_2)(T(\bm{x})) \\
              &= \lambda_1(T(\bm{x}))+\lambda_2(T(\bm{x})) \\
              &= (\lambda_1T)(\bm{x})+(\lambda_2T)(\bm{x}) \\
              &= (\lambda_1T+\lambda_2T)(\bm{x})
            \end{align*}
            より,$(\lambda_1+\lambda_2)T=\lambda_1T+\lambda_2T$.
          \item $\lambda \in K, T_1,T_2 \in \mathcal{L}(V_1,V_2)$ とおくと,$\bm{x} \in V_1$ に対し,
            \begin{align*}
              (\lambda(T_1+T_2))(\bm{x}) &= \lambda((T_1+T_2)(\bm{x})) \\
              &= \lambda(T_1(\bm{x})+T_2(\bm{x})) \\
              &= \lambda T_1(\bm{x})+\lambda T_2(\bm{x}) \\
              &= (\lambda T_1)(\bm{x})+(\lambda T_2)(\bm{x}) \\
              &= (\lambda T_1 + \lambda T_2)(\bm{x})
            \end{align*}
            より,$\lambda(T_1+T_2)=\lambda T_1 + \lambda T_2$.
          \item $\lambda_1,\lambda_2 \in K, T \in \mathcal{L}(V_1,V_2)$ とおくと,$\bm{x} \in V_1$ に対し,
            \begin{align*}
              ((\lambda_1\lambda_2)T)(\bm{x}) &= (\lambda_1\lambda_2)(T(\bm{x})) \\
              &= \lambda_1(\lambda_2(T(\bm{x}))) \\
              &= \lambda_1((\lambda_2 T)(\bm{x})) \\
              &= (\lambda_1(\lambda_2 T))(\bm{x})
            \end{align*}
            より,$(\lambda_1\lambda_2)T=\lambda_1(\lambda_2 T)$.
          \item $T \in \mathcal{L}(V_1,V_2), \bm{x} \in V_1$ に対し,
            \begin{align*}
              (1T)(\bm{x}) &= 1(T(\bm{x})) \\
              &= T(\bm{x})
            \end{align*}
            より,$1T=T$.
        \end{enumerate}
      \end{proof}
      \begin{prop}\label{composition-of-linear-map-is-linear-map}
        $V_1,V_2,V_3$: $K$ 上のベクトル空間,$T_1 \in \mathcal{L}(V_1,V_2)$,$T_2 \in \mathcal{L}(V_2,V_3)$.このとき,$T_2 \circ T_1 \in \mathcal{L}(V_1,V_3)$.
      \end{prop}
      \begin{proof}
        $\lambda_1,\lambda_2 \in K, \bm{x}_1,\bm{x}_2 \in V_1$ に対し,
        \begin{align*}
          (T_2 \circ T_1)(\lambda_1\bm{x}_1+\lambda_2\bm{x}_2) &= T_2(T_1(\lambda_1\bm{x}_1+\lambda_2\bm{x}_2)) \\
          &= T_2(\lambda_1T_1(\bm{x}_1)+\lambda_2T_1(\bm{x}_2)) \\
          &= \lambda_1T_2(T_1(\bm{x}_1))+\lambda_2T_2(T_1(\bm{x}_2)) \\
          &= \lambda_1(T_2 \circ T_1)(\bm{x}_1)+\lambda_2(T_2 \circ T_1)(\bm{x}_2).
        \end{align*}
      \end{proof}
      \begin{sym}
        命題 \ref{composition-of-linear-map-is-linear-map} における $T_2 \circ T_1$ を $T_2T_1$ と書く. 
      \end{sym}
      \begin{prop}
        以下が成り立つ.
        \begin{enumerate}
          \item $V_1,V_2,V_3$: ベクトル空間,$f \in \mathcal{L}(V_1,V_2)$,$0 \in \mathcal{L}(V_2,V_3)$: $0$ 写像.このとき,$0f=0 \in \mathcal{L}(V_1,V_3)$.
          \item $V_1,V_2,V_3$: ベクトル空間,$0 \in \mathcal{L}(V_1,V_2)$: $0$ 写像,$f \in \mathcal{L}(V_2,V_3)$.このとき,$f0=0 \in \mathcal{L}(V_1,V_3)$.
          \item $V_1,V_2$: ベクトル空間,$f \in \mathcal{L}(V_1,V_2)$.このとき,$1_{V_2}f=f1_{V_1}=f$.
        \end{enumerate}
      \end{prop}
      \begin{proof}
        \mbox{}
        \begin{enumerate}
          \item $\bm{x} \in V_1$ に対し,$(0f)(\bm{x})=0(f(\bm{x}))=\bm{0}=0(\bm{x})$.
          \item $\bm{x} \in V_1$ に対し,$(f0)(\bm{x})=f(0(\bm{x}))=f(\bm{0})=\bm{0}=0(\bm{x})$.
          \item $\bm{x} \in V_1$ に対し,$(1_{V_2}f)(\bm{x})=1_{V_2}(f(\bm{x}))=f(\bm{x})$,$(f1_{V_1})(\bm{x})=f(1_{V_1}(\bm{x}))=f(\bm{x})$.
        \end{enumerate}
      \end{proof}
      \begin{prop}
        以下が成り立つ.
        \begin{enumerate}
          \item $V_1,V_2,V_3$: $K$ 上のベクトル空間,$f \in \mathcal{L}(V_1,V_2)$,$g \in \mathcal{L}(V_2,V_3)$,$\lambda \in K$.このとき,$g(kf)=(kg)f=k(gf)$.
          \item $V_1,V_2,V_3$: ベクトル空間,$f \in \mathcal{L}(V_1,V_2)$,$g_1, g_2 \in \mathcal{L}(V_2,V_3)$.このとき,$(g_1+g_2)f=g_1f+g_2f$.
          \item $V_1,V_2,V_3$: ベクトル空間,$f_1,f_2 \in \mathcal{L}(V_1,V_2)$,$g \in \mathcal{L}(V_2,V_3)$.このとき,$g(f_1+f_2)=gf_1+gf_2$.
          \item $V_1,V_2,V_3,V_4$: ベクトル空間,$f_1 \in \mathcal{L}(V_1,V_2), f_2 \in \mathcal{L}(V_2,V_3), f_3 \in \mathcal{L}(V_3,V_4)$.このとき,$(f_3f_2)f_1=f_3(f_2f_1)$.
        \end{enumerate}
      \end{prop}
      \begin{proof}
        \mbox{}
        \begin{enumerate}
          \item $\bm{x} \in V_1$ に対し,
            \begin{align*}
              (g(kf))(\bm{x}) &= g((kf)(\bm{x}) \\
              &= g(k(f(\bm{x}))) \\
              &= k(g(f(\bm{x}))) \\
              &= k((gf)(\bm{x})),\\
              (kg)(f)(\bm{x}) &= (kg)(f(\bm{x})) \\
              &= k(g(f(\bm{x}))) \\
              &= k((gf)(\bm{x}))
            \end{align*}
            よって $g(kf)=(kg)f=k(gf)$.
          \item $\bm{x} \in V_1$ に対し,
            \begin{align*}
              ((g_1+g_2)f)(\bm{x}) &= (g_1+g_2)(f(\bm{x})) \\
              &= g_1(f(\bm{x}))+g_2(f(\bm{x})) \\
              &= (g_1f)(\bm{x})+(g_2f)(\bm{x}) \\
              &= (g_1f+g_2f)(\bm{x})
            \end{align*}
            よって $(g_1+g_2)f=g_1f+g_2f$.
          \item $\bm{x} \in V_1$ に対し,
            \begin{align*}
              (g(f_1+f_2))(\bm{x}) &= g((f_1+f_2)(\bm{x})) \\
              &= g(f_1(\bm{x})+f_2(\bm{x})) \\
              &= g(f_1(\bm{x}))+g(f_2(\bm{x})) \\
              &= (gf_1)(\bm{x})+(gf_2)(\bm{x}) \\
              &= (gf_1+gf_2)(\bm{x})
            \end{align*}
            よって $g(f_1+f_2)=gf_1+gf_2$.
          \item $\bm{x} \in V_1$ に対し,
            \begin{align*}
               ((f_3f_2)f_1)(\bm{x}) &= (f_3f_2)(f_1(x)) \\
               &= f_3(f_2(f_1(\bm{x}))) \\
               &= f_3((f_2f_1)(\bm{x}) \\
               &= (f_3(f_2f_1))(\bm{x})
            \end{align*}
            よって $(f_3f_2)f_1=f_3(f_2f_1)$.
        \end{enumerate}
      \end{proof}
      \begin{prop}\label{image-and-preimage-of-subspace-is-subspace}
        $V_1, V_2$: $K$ 上のベクトル空間,$T \in \mathcal{L}(V_1,V_2)$.このとき,以下が成り立つ.
        \begin{enumerate}
          \item $W \subseteq V_1$: 部分空間に対し,$T(W)=\{T(\bm{x}) \mid \bm{x} \in W\} \subseteq V_2$ は $V_2$ の部分空間.
          \item $W \subseteq V_2$: 部分空間に対し,$T^{-1}(W)=\{\bm{x} \in V_1 \mid T(\bm{x}) \in W\} \subseteq V_1$ は $V_1$ の部分空間.
        \end{enumerate}
      \end{prop}
      \begin{proof}
        \mbox{}
        \begin{enumerate}
          \item $\lambda_1,\lambda_2 \in K, \bm{y}_1,\bm{y}_2 \in T(W)$ とおくと,$\exists \bm{x}_1,\bm{x}_2 \in W: T(\bm{x}_1)=\bm{y}_1,T(\bm{x}_2)=\bm{y}_2$.このとき,$\lambda_1\bm{x}_1+\lambda_2\bm{x}_2 \in W$ より $\lambda_1\bm{y}_1+\lambda_2\bm{x}_2=\lambda_1T(\bm{x}_1)+\lambda_2T(\bm{x}_2)=T(\lambda_1\bm{x}_1+\lambda_2\bm{x}_2) \in T(W)$.
          \item $\lambda_1,\lambda_2 \in K, \bm{x}_1,\bm{x}_2 \in T^{-1}(W)$ とおくと,$T(\bm{x}_1),T(\bm{x}_2) \in W$ より $T(\lambda_1\bm{x}_1+\lambda_2\bm{x}_2)=\lambda_1T(\bm{x}_1)+\lambda_2T(\bm{x}_2) \in W$ であるから,$\lambda_1\bm{x}_1+\lambda_2\bm{x}_2 \in W$.
        \end{enumerate}
      \end{proof}  
      \begin{dfn}
        $V_1, V_2$: ベクトル空間,$T \in \mathcal{L}(V_1,V_2)$.
        \begin{enumerate}
          \item $\Ker T := \{\bm{x} \in V_1 \mid T(\bm{x})=\bm{0}\} = T^{-1}(\{\bm{0}\})$ を $T$ の核という.
          \item $\Img T := \{T(\bm{x}) \mid \bm{x} \in V_1\} = T(V_1)$ を $T$ の像という.
        \end{enumerate}
      \end{dfn}
      \begin{cor}
        $V_1,V_2$: ベクトル空間,$T \in \mathcal{L}(V_1,V_2)$.
        \begin{enumerate}
          \item $\Ker T \subseteq V_1$ は $V_1$ の部分空間.
          \item $\Img T \subseteq V_2$ は $V_2$ の部分空間.
        \end{enumerate}
      \end{cor}
      \begin{proof}
        命題 \ref{image-and-preimage-of-subspace-is-subspace} から従う.
      \end{proof}
\end{document}